\usepackage{booktabs}
\usepackage{subfigure}
\usepackage{color, colortbl}
\definecolor{Gray}{gray}{0.9}

\def\begincols{\begin{columns}}
\def\begincol{\begin{column}}
\def\endcol{\end{column}}
\def\endcols{\end{columns}}

\setbeamertemplate{caption}[numbered]
\setbeamertemplate{itemize items}[default]
\setbeamertemplate{itemize subitem}[circle]

% \begincols
% \begincol{.48\textwidth}
% \endcol
% \begincol{.48\textwidth}
% \endcol
% \endcols

% How many criminal candidates were there? Where and when?
% 6,499 out of 35,075 contesting candidates -- close to 20\%

% Did criminal candidates win elections? Where and when?
% Given that a candidate is a criminal, the probability of winning the election is about 0.22; in contrast, Given that a candidate is NOT a criminal, the probability of winning the election is about 0.12. (X)

% How many criminal politicians were there? Where and when? 

% Party affiliations: BJP (23\%), Congress (18\%), 
